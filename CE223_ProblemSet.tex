% Options for packages loaded elsewhere
\PassOptionsToPackage{unicode}{hyperref}
\PassOptionsToPackage{hyphens}{url}
%
\documentclass[
]{article}
\usepackage{amsmath,amssymb}
\usepackage{lmodern}
\usepackage{iftex}
\ifPDFTeX
  \usepackage[T1]{fontenc}
  \usepackage[utf8]{inputenc}
  \usepackage{textcomp} % provide euro and other symbols
\else % if luatex or xetex
  \usepackage{unicode-math}
  \defaultfontfeatures{Scale=MatchLowercase}
  \defaultfontfeatures[\rmfamily]{Ligatures=TeX,Scale=1}
\fi
% Use upquote if available, for straight quotes in verbatim environments
\IfFileExists{upquote.sty}{\usepackage{upquote}}{}
\IfFileExists{microtype.sty}{% use microtype if available
  \usepackage[]{microtype}
  \UseMicrotypeSet[protrusion]{basicmath} % disable protrusion for tt fonts
}{}
\makeatletter
\@ifundefined{KOMAClassName}{% if non-KOMA class
  \IfFileExists{parskip.sty}{%
    \usepackage{parskip}
  }{% else
    \setlength{\parindent}{0pt}
    \setlength{\parskip}{6pt plus 2pt minus 1pt}}
}{% if KOMA class
  \KOMAoptions{parskip=half}}
\makeatother
\usepackage{xcolor}
\IfFileExists{xurl.sty}{\usepackage{xurl}}{} % add URL line breaks if available
\IfFileExists{bookmark.sty}{\usepackage{bookmark}}{\usepackage{hyperref}}
\hypersetup{
  hidelinks,
  pdfcreator={LaTeX via pandoc}}
\urlstyle{same} % disable monospaced font for URLs
\usepackage{color}
\usepackage{fancyvrb}
\newcommand{\VerbBar}{|}
\newcommand{\VERB}{\Verb[commandchars=\\\{\}]}
\DefineVerbatimEnvironment{Highlighting}{Verbatim}{commandchars=\\\{\}}
% Add ',fontsize=\small' for more characters per line
\newenvironment{Shaded}{}{}
\newcommand{\AlertTok}[1]{\textcolor[rgb]{1.00,0.00,0.00}{\textbf{#1}}}
\newcommand{\AnnotationTok}[1]{\textcolor[rgb]{0.38,0.63,0.69}{\textbf{\textit{#1}}}}
\newcommand{\AttributeTok}[1]{\textcolor[rgb]{0.49,0.56,0.16}{#1}}
\newcommand{\BaseNTok}[1]{\textcolor[rgb]{0.25,0.63,0.44}{#1}}
\newcommand{\BuiltInTok}[1]{#1}
\newcommand{\CharTok}[1]{\textcolor[rgb]{0.25,0.44,0.63}{#1}}
\newcommand{\CommentTok}[1]{\textcolor[rgb]{0.38,0.63,0.69}{\textit{#1}}}
\newcommand{\CommentVarTok}[1]{\textcolor[rgb]{0.38,0.63,0.69}{\textbf{\textit{#1}}}}
\newcommand{\ConstantTok}[1]{\textcolor[rgb]{0.53,0.00,0.00}{#1}}
\newcommand{\ControlFlowTok}[1]{\textcolor[rgb]{0.00,0.44,0.13}{\textbf{#1}}}
\newcommand{\DataTypeTok}[1]{\textcolor[rgb]{0.56,0.13,0.00}{#1}}
\newcommand{\DecValTok}[1]{\textcolor[rgb]{0.25,0.63,0.44}{#1}}
\newcommand{\DocumentationTok}[1]{\textcolor[rgb]{0.73,0.13,0.13}{\textit{#1}}}
\newcommand{\ErrorTok}[1]{\textcolor[rgb]{1.00,0.00,0.00}{\textbf{#1}}}
\newcommand{\ExtensionTok}[1]{#1}
\newcommand{\FloatTok}[1]{\textcolor[rgb]{0.25,0.63,0.44}{#1}}
\newcommand{\FunctionTok}[1]{\textcolor[rgb]{0.02,0.16,0.49}{#1}}
\newcommand{\ImportTok}[1]{#1}
\newcommand{\InformationTok}[1]{\textcolor[rgb]{0.38,0.63,0.69}{\textbf{\textit{#1}}}}
\newcommand{\KeywordTok}[1]{\textcolor[rgb]{0.00,0.44,0.13}{\textbf{#1}}}
\newcommand{\NormalTok}[1]{#1}
\newcommand{\OperatorTok}[1]{\textcolor[rgb]{0.40,0.40,0.40}{#1}}
\newcommand{\OtherTok}[1]{\textcolor[rgb]{0.00,0.44,0.13}{#1}}
\newcommand{\PreprocessorTok}[1]{\textcolor[rgb]{0.74,0.48,0.00}{#1}}
\newcommand{\RegionMarkerTok}[1]{#1}
\newcommand{\SpecialCharTok}[1]{\textcolor[rgb]{0.25,0.44,0.63}{#1}}
\newcommand{\SpecialStringTok}[1]{\textcolor[rgb]{0.73,0.40,0.53}{#1}}
\newcommand{\StringTok}[1]{\textcolor[rgb]{0.25,0.44,0.63}{#1}}
\newcommand{\VariableTok}[1]{\textcolor[rgb]{0.10,0.09,0.49}{#1}}
\newcommand{\VerbatimStringTok}[1]{\textcolor[rgb]{0.25,0.44,0.63}{#1}}
\newcommand{\WarningTok}[1]{\textcolor[rgb]{0.38,0.63,0.69}{\textbf{\textit{#1}}}}
\usepackage{graphicx}
\makeatletter
\def\maxwidth{\ifdim\Gin@nat@width>\linewidth\linewidth\else\Gin@nat@width\fi}
\def\maxheight{\ifdim\Gin@nat@height>\textheight\textheight\else\Gin@nat@height\fi}
\makeatother
% Scale images if necessary, so that they will not overflow the page
% margins by default, and it is still possible to overwrite the defaults
% using explicit options in \includegraphics[width, height, ...]{}
\setkeys{Gin}{width=\maxwidth,height=\maxheight,keepaspectratio}
% Set default figure placement to htbp
\makeatletter
\def\fps@figure{htbp}
\makeatother
\setlength{\emergencystretch}{3em} % prevent overfull lines
\providecommand{\tightlist}{%
  \setlength{\itemsep}{0pt}\setlength{\parskip}{0pt}}
\setcounter{secnumdepth}{-\maxdimen} % remove section numbering
\ifLuaTeX
  \usepackage{selnolig}  % disable illegal ligatures
\fi

\author{}
\date{}

\begin{document}

\hypertarget{header-n0}{%
\section{eExercise 1}\label{header-n0}}

1.1

\begin{enumerate}
\def\labelenumi{\arabic{enumi}.}
\item
  Find the solution to the following first-order differential equations:
\end{enumerate}

\[y'+y = e^{-2x}\]

\[xy'-2y = x^4\]

\textbf{Answer}

1.( a )

\[y' + y = e^{-2x}\]

Solution to homogenous{[}齐次{]} pat,

\begin{quote}
先求齐次式, 求不定积分
\end{quote}

\[y'+y=0\]

is

\[\frac{dy}{dx} + y =0\]

\[\frac{dy}{dx} =- y\]

\[\frac{dy}{y} =- dx\]

\[\frac{dy}{y} + dx =0\]

\[\frac{dy}{y}+dx = 0\]

\[\int {\frac{dy}{y}} + \int dx = \int0\]

\[lny+x=lnk\]

\[ln\frac{y}{k}=-x\]

\[y=ke^{-x}\]

Solution to nonhomogenous e.g. found using variation method

\[y = K(x)e^{-x}\]

\begin{quote}
齐次方程通解, 通解带入原式, (前面求导+后面求导), 然后积分求出 \(K(x)\)
\end{quote}

Thus
\includegraphics{D:/W_WORKSPACE/E_ESSEX/CE223/CE223_MD/CE223_ProblemSet.assets/image-20210412084841509.png}

\[y' + y = e^{-2x}\]

\[[K(x)e^{-x}]' + K(x)e^{-x} = e^{-2x}\]

\[[K(x)]'e^{-x} + K(x)[e^{-x}]'  + K(x)e^{-x} = e^{-2x}\]

\[K'(x)e^{-x}- K(x)e^{-x} + K(x)e^{-x} = e^{-2x}\]

\[K'(x) = e^{-x}\]

\[K(x) = \int e^{-x}dx+k\]

Therefore

\begin{quote}
\(K(x)\) 代回原式
\end{quote}

\[K(x)=-e^{-x}+ k\]

Hence, the solution to (1) is

\[y = (-e^{-x}+k)e^{-x}=-e^{-2x}+ke^{-x} ,(Check)\]

1.( b )

\[xy' - 2y =x^4\]

Solutions to homogeneous equations

\begin{quote}
求齐次方程通解
\end{quote}

\[xy' - 2y = 0\]

\[x · \frac{dy}{dx} - 2y = 0\]

\[x · \frac{dy}{dx}  = 2y\]

\[x · dy =  2y · dx\]

\[\frac{dy}{y}= 2\frac{dx}{x}\]

\[\int \frac{dy}{y} = 2 \int{\frac{dx}{x} + \int 0 }\]

\[ln y = 2lnx + ln k\]

\[ln y = ln x^2 + lnk\]

\[ln y = ln ( k·x^2 )\]

\[y = kx^2\]

Solution to nonhomogenous e.g. found using variation method

\[y = K(x)x^2\]

Using parameter variation method, the solution to nonhomogenous e.g. is
found

\begin{quote}
齐次方程通解, 通解带入原式, (前面求导+后面求导), 然后积分求出 \(K(x)\)
\end{quote}

\[xy' - 2y =x^4\]

\[x [K(x)x^2]' -2 ·K(x)x^2 = x^4\]

\[x \{ [K(x)]'x^2 + K(x)[x^2]'\} -2 ·K(x)x^2 = x^4\]

\[x \{ K'(x)x^2 + K(x)2x\} -2 ·K(x)x^2 = x^4\]

\[K'(x)x^3 + 2·K(x)x^2 -2 ·K(x)x^2 = x^4\]

\[K'(x) x^3 = x^4\]

\[K'(x) = x\]

\[\int K'(x) =\int x+ \int0\]

\[K(x) = \frac{1}{2} x^2 + k\]

Thus

\[y = K(x) x^2\]

\[y =  (\frac{1}{2} x^2 + k ) x^2\]

\[y =  \frac{1}{2} x^4 + k·x^2, (Check)\]

\begin{enumerate}
\def\labelenumi{\arabic{enumi}.}
\item
  Verify that \(y = e^{−x}\) is a solution to the differential equation
  \(y''+2y'+y = 0\). What is special about this equation? Use method of
  parameter variation to find the second solution.
\end{enumerate}

\textbf{Answer}

To verify \(y = e^{−x}\)

\[y ' = -e^{-x}\]

\[y'' = e^{-x}\]

Thus,

\[y''+2y'+y = 0\]

\[e^{-x} + 2(-e^{-x})+e^{-x} = 0\]

The special thing about the differential equation is that its
characteristic e.g. has repeat roots:

\[\lambda^2 + 2\lambda+1 = 0\]

Parameters variations method can be applied to

\begin{figure}
\centering
\includegraphics{D:/W_WORKSPACE/E_ESSEX/CE223/CE223_MD/CE223_ProblemSet.assets/image-20210412111118727.png}
\caption{}
\end{figure}

\[y = K(x)e^{-x}\]

Substitution in the differential equation to obtains

\[K''(x)e^{-x}-K'(x)e^{-x}-K'(x)e^{-x}+K(x)e^{-x}+2[K'(x)e^{-x} - K(x)e^{-x}] + K(x)e^{-x} = 0\]

\[K''(x)e^{-x} = 0\]

\[K''(x) = 0\]

\[K'(x) = A , (constant)\]

\[K(x) = Ax + B\]

Thus,

\[y = (Ax +B) e^{-x}\]

\[y = Axe^{-x}_{\ \ \ \ second\ from P_oV_o} + Be^{-x}_{\ \ \ \ original }, complete\ solution\]

\begin{enumerate}
\def\labelenumi{\arabic{enumi}.}
\item
  Verify that \(y = −cos(x)\) is a solution to the differential equation
  \(y''−y = 2cos(x)\). Find the general solution of the equation.
\end{enumerate}

\begin{quote}
\begin{Shaded}
\begin{Highlighting}[]
\NormalTok{ line:     }\FunctionTok{\textbackslash{}ldots} 
\NormalTok{ diagonal: }\FunctionTok{\textbackslash{}ddots} 
\NormalTok{ vertical: }\FunctionTok{\textbackslash{}vdots}
\end{Highlighting}
\end{Shaded}

\[line \ldots\]

\[diagonal \ddots\]

\[vertical \vdots\]

\begin{figure}
\centering
\includegraphics{D:/W_WORKSPACE/E_ESSEX/CE223/CE223_MD/CE223_ProblemSet.assets/UBhLB.jpg}
\caption{}
\end{figure}
\end{quote}

\textbf{Answer}

Considering \(y=-\cos(x)\)

\begin{pmatrix}

-3a_1 + 2 a_0 = 0 
\\ 2a_1=4
\\ 2b = 1
\\ -2c_1 + bc_2 = 1
\\ -2c_2 -bc_1 = 0 

\end{pmatrix}

\rightarrow

\begin{Bmatrix}




\end{Bmatrix}

\begin{enumerate}
\def\labelenumi{\arabic{enumi}.}
\item
  Find the particular solution{[}特解{]} of the differential equation
  \(y''−3y'+2y = 4x + e^{3x} + sin(2x)\) . What is the general solution
  to the equation?
\end{enumerate}

\textbf{Answer}

The particular solution, \(y_p\), has an expression ( shape ), similar
to the excitation function, i.e.

\[y_p =a_0 + a_1x\]

\begin{enumerate}
\def\labelenumi{\arabic{enumi}.}
\item
  Find the solution to \(y'' + y = sin(5t)\) . If the input to this
  differential equation, \(sin(5t)\), (providing the forced oscillation)
  is replaced by \(sin(t)\), find the solution.

  Note:

  \[y' = \frac{d}{dx}y \ and \ y'' = \frac{d^2}{dx^2}y\]

  (derivatives are with respect to x)
\end{enumerate}

\[y' =\frac{d}{dt}y\ and\ y''=\frac{d^2}{dt^2}y\]

(derivatives are with respect to t)

\textbf{Answer}

\hypertarget{header-n148}{%
\section{Exercise 2}\label{header-n148}}

\hypertarget{header-n149}{%
\subsubsection{P1 Find the relationship between the output voltage and
input voltage in the following circuits both in the time-domain and in
the Laplace-domain. Assuming the initial conditions for C and L are zero
and the excitation is impulsive find the behaviour of the output
voltage.}\label{header-n149}}

\hypertarget{header-n150}{%
\subsubsection{P2 If the impulsive excitation is replaced by a
sinusoidal excitation, find the frequency domain voltage transfer
function for each case.}\label{header-n150}}

\hypertarget{header-n151}{%
\subsubsection{P5 Then find the mag. and phase of each transfer
function.}\label{header-n151}}

\begin{figure}
\centering
\includegraphics{D:/W_WORKSPACE/E_ESSEX/CE223/CE223_MD/CE223_ProblemSet.assets/image-20210412171719058.png}
\caption{}
\end{figure}

\begin{figure}
\centering
\includegraphics{D:/W_WORKSPACE/E_ESSEX/CE223/CE223_MD/CE223_ProblemSet.assets/image-20210412171807625.png}
\caption{}
\end{figure}

\textbf{Answer:}

\begin{figure}
\centering
\includegraphics{D:/W_WORKSPACE/E_ESSEX/CE223/CE223_MD/CE223_ProblemSet.assets/image-20210529205254309.png}
\caption{}
\end{figure}

writing voltage around the

\begin{quote}
电容是 1/C 积分 i(t)
\end{quote}

\[+ R_i(t) + \frac{1}{C}\int i(t)dt +R_2 i(t) = V_i(t)\]

\[-V_i(t) + R_i(t) + \frac{1}{C}\int i(t)dt +R_2 i(t) = 0\]

where

\[i(t) =\frac{V_2(t)}{R_2}\]

Differentiate
\(-V_i(t) + R_i(t) + \frac{1}{C}\int i(t)dt +R_2 i(t) = 0\) with respect
to i

\begin{quote}
同时求导
\end{quote}

\[-\frac{d V_i(t)}{dt} + R_1 \frac{d i(t)}{dt} + \frac{1}{C}i(t) + R_2 \frac{d i(t) }{dt} = 0\]

\hypertarget{header-n169}{%
\subsubsection{P2 If the impulsive excitation is replaced by a
sinusoidal excitation, find the frequency domain voltage transfer
function for each case.}\label{header-n169}}

\begin{figure}
\centering
\includegraphics{D:/W_WORKSPACE/E_ESSEX/CE223/CE223_MD/CE223_ProblemSet.assets/image-20210530123836468.png}
\caption{}
\end{figure}

\[i(t) = i_1(t) +i_2(t)\]

Therefore,

\[i(t)\]

\hypertarget{header-n177}{%
\subsubsection{P5 Then find the mag. and phase of each transfer
function.}\label{header-n177}}

\begin{figure}
\centering
\includegraphics{D:/W_WORKSPACE/E_ESSEX/CE223/CE223_MD/CE223_ProblemSet.assets/image-20210530123513220.png}
\caption{}
\end{figure}

This is from the circuit

\[i(t) = i_1(t)+i_2(t)\]

therefore

\begin{quote}
并联 1/R, 1/L,C
\end{quote}

\[i(t) = \frac{V_o(t)}{R}+C\frac{dV_o(t)}{dt}\]

\# Exercise 3

\hypertarget{header-n201}{%
\section{Exercise 4}\label{header-n201}}

\end{document}
